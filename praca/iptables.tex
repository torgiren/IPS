\chapter{Iptables}
	\section{Zarys ogólny}
		Iptables jest konsolowym interfejsem dla netfilter-a. Pozwala on na tworzenie łańcuchów, dodawanie oraz usuwanie reguł, oraz wyświetlanie statystyk.
		Często nazwa iptables błędnie używana jest wymiennie z netfilter. Wynika to z faktu braku innych interfejsów do obsługi netfilter.
	\section{Polecenia iptables}
		Najczęściej wykorzystywane polecenia iptables to:
		\begin{description}
			\item[-t \param{tablica}] \hfill \\
				opcjonalny parametr, który możemy przekazać do każdej poniżej opisanej opcji, definiujący tablicę na której będziemy wykonywać operacje.
				Jeżeli ten parametr nie zostanie zdefiniowany, domyślną tablicą jest tablica \textit{filter}.
			\item[-A \param{łańcuch} \param{reguła}] \hfil \\
				dodawanie nowej reguły na koniec łańcucha.
			\item[-I \param{łańcuch} {[nr]} \param{reguła}] \hfill \\
				wstawienie nowej reguły do łańcucha. Jeżeli zostanie podany parametr \textit{nr}, reguła zostaje wstawiona na pozycję \textit{nr}.
				Jeżeli parametr nie zostanie podany, domyślną wartością jest 1, czyli początek łańcucha.
			\item[-D \param{łańcuch} \param{reguła}] \hfill \\
				usuwa z łańcucha regułę podaną przez specyfikację.
			\item[-D \param{łańcuch} \param{nr}] \hfill \\
				usuwa z łańcucha regułę podaną przez numer porządkowy, liczony od 1.
			\item[-N \param{łańcuch}] \hfill \\
				tworzy nowy łańcuch o nazwie \textit{łańcuch}.
			\item[-F {[\textit{łańcuch}]}] \hfill \\
				usuwa wszystkie reguły z zadanego łańcucha. Jeżeli nie zostanie podany łańcuch, wyczyszczone zostaną wszystkie łańcuchy.
			\item[-X \param{łańcuch}] \hfill \\
				usuwa zadany łańcuch. Aby móc usunąć łańcuch, musi on być wcześniej wyczyszczony.	
			\item[-P \param{łańcuch} \param{polityka}] \hfill \\
				ustawia politykę dla łańcucha.
			\item[-L {[\textit{łańcuch}]}] \hfill \\
				wypisuje reguły w łańcuchu. Jeżeli wartość \textit{łańcuch} nie zostanie podana, zostają wypisany wszystkie łańcuchy.\\
				Często używane opcje polecenia -L, to:
				\begin{description}
					\item[-n] \hfill \\
						nie zamienia adresów ip na nazwy domenowe - często przyśpiesza wypisywanie wyników, gdyż nie oczekujemy na odpowiedzi od revdns-a.
					\item[-v] \hfill \\
						tryb gadatliwy. Wypisuje statystyki ilościowe i objętościowe dla wypisywanych reguł
				\end{description}
		\end{description}


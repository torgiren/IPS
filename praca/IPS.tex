\documentclass[a4paper,10pt]{book}
\usepackage[utf8]{inputenc}
\usepackage{polski}
\title{System IPS opaty o iptables}
\author{Marcin TORGiren Fabrykowski}
\setcounter{secnumdepth}{4}
\setcounter{tocdepth}{4}
\begin{document}
\maketitle
\newpage
\tableofcontents
\chapter*{Wstęp}
\addcontentsline{toc}{chapter}{Wstęp}
\chapter{Opis problemu}
	\section{O IPS}
		\subsection{Co to jest IPS}
		System IPS (ang. Intrusion Prevention System) jest to system wykrywania i blokowania ataków sieciowych.
		Jego zadanie polega na analizie ruchu sieciowego wchodzącego do oraz przechodzącego przez niego oraz odpowiednie reagowanie w przypadku wykrycia nienormalnych zachowań sieci.
		\subsection{Schemat działania}
			\begin{enumerate}
				\item Analiza ruchu sieciowego
				\item Wykrycie zachowań pasujących do zdefiniowanych reguł bezpieczeństwa
				\item Reakcja na wykryte niebezpieczne zachowanie
				\item Poinformowanie administratora o próbie ataku oraz podjętych działaniach
				\item Zapisanie danych o ataku oraz podjętych działaniach do bazy danych
				\item Udostępnienie administratorowi wglądu w historię ataków
			\end{enumerate}
	\section{Analiza ruchu sieciowego}
		\subsection{Używane narzędzie}
			Jako systemu analizującego ruch sieciowy wykorzystam pakiet Netfilter konfigurowany za pomocą iptables.
		\subsection{Netfilter}
			\subsubsection{Ogólny zarys}
				Netfilter jest oprogramowaniem pozwalającym na filtrowanie pakietów, ich translacje (NAT) oraz inne manipulację.
				Od wersji jądra 2.4.x, pakiet netfilter jest umieszczony wewnątrz niego.
				Potrafi on dopasowywać analizowane pakiety ze względu na szeroką gamę kryteriów, jak również przeprowadzić szereg operacji na danych pakietach.
			\subsubsection{Najważniejsze kryteria dopasowania}
				Netfilter bogaty wachlarz możliwości dopasowywania pakietów
				\begin{description}
					\item[-{}-source, -{}-src, -s \textit{\textless adres \textgreater} ] \hfill \\
						dopasowuje adres źródłowy pakietu do podanego jako \textit{adres}
					\item[-{}-destination, -{}- dst, -d \textit{\textless adres \textgreater}] \hfill\\
						dopasowuje adres docelowy pakietu do podanego jako \textit{adres}
					\item[-{}-protocol, -p \textit{\textless protocol \textgreater}] \hfill \\
						dopasowuje protokół używany przez pakiet.\\
						Najczęściej używane protokołu to: \textit{tcp},\textit{udp},\textit{icmp}
					\item[-{}-source-port, -{}-sport \textit{\textless port \textgreater} ]\hfill\\
						dopasowuje port źródłowy pakietu.\\
						Aby użyć tego dopasowania należy zdefiniować protokół.
					\item[-{}-destination-port, -{}-dport \textit{\textless port \textgreater}] \hfill \\
						dopasowuje port docelowy pakietu.\\
						Aby użyć tego dopasowania należy zdefiniować protokół.
				\end{description}
				\begin{itemize}
					\item flagi TCP
					\item adres MAC
					\item interface sieciowy
					\item czas i data
					\item ilość pakietów w danym czasie
				\end{itemize}
			\subsubsection{Najważniejsze działania}
				\begin{itemize}
					\item zaakceptowanie
					\item odrzucenie (DROP, REJECT)
					\item logowanie pakietu
					\item zmiana wartości ttl
					\item zmiana wartości portu
					\item zmiana wartości adresu
				\end{itemize}
\end{document}

\chapter*{Podsumowanie}
	\addcontentsline{toc}{chapter}{Podsumowanie}
	Przedstawione metody ataku stanowią duże zagrożenie dla systemów informatycznych, jednak jak udało się pokazać, obrona przed większością z nich nie jest trudna.
	Większość przedstawionych ataków sieciowych można zablokować wykorzystując jedną regułę zapory ogniowej, co nie niesie dużego narzutu czasu procesora, a jest w stanie uchronić broniony serwer przed atakami a właściciela przed stratami.\\
	Opracowany w niniejszej pracy system IPS jest w stanie zabezpieczyć serwer przed omówionymi atakami.
	Oczywiście trzeba mieć na uwadze, że przedstawione ataki są atakami głównie w trzeciej warstwie modelu OSI, dlatego przedstawiony system w żadnym wypadku nie chroni serwera przed atakami z warstwy aplikacji - takimi jak aktywność wirusów bądź błędy w skryptach PHP, jednakże pełni on ważną rolę w obronie systemu.
	

\chapter{Raportowanie}
	\section{Logowanie do pliku}
		Logowanie pakietów przy użyciu opcji \textit{-j LOG}, w domyślnej konfiguracji zapisuje pakiety do pliku \textit{/var/log/messages}.
		W tym pliku zapisywane są wszystkie zdarzenia logowane na komputerze.
		Mnogość logowanych zdarzeń może utrudniać filtrowanie interesujących nas informacji z systemu IPS.
		Dlatego też, należy przekierować zapisywanie zapisów do osobnego pliku.
		W systemie Ubuntu 10.4, domyślnym systemem logowania jest \textit{rsyslog}, dlatego przedstawiona zostanie konfiguracja dla tego systemu.
		
		Cechą wspólną wszystkich zapisów tworzonych przez opisywany system IPS, jest występowanie frazy "IPS:".
		Wszystkie wpisy zawierające powyższą frazę traktujemy jako wpis systemu i przekierowujemy do osobnego pliku.
		Aby to osiągnąć, umieszczamy plik konfiguracyjny:
		\lstinputlisting[caption=/etc/rsyslog.d/ips.conf,basicstyle=\footnotesize,numbers=left]{../reports/conf/etc/rsyslog.d/ips.conf}
	\section{Logowanie do bazy danych}
		Ponieważ operowanie na plikach tekstowych jest zarówno niewygodne jak i mniej optymalne niż operowanie na rekordach bazy danych, zdarzenia będą zapisywane do bazy danych.

		Zapisywania będziemy dokonywać, uruchamiając co minutę skrypt napisany w języku Python.
		Skrypt ten będzie filtrować plik \textit{/var/log/ips.log}, zawierający zdarzenia z systemu IPS, tak, aby uzyskać tylko zdarzenia zarejestrowane w poprzedniej minucie.
		Następnie przeprowadzi analizę wpisów zdarzeń i dokona wpisu do bazy danych.

		Każdy rekord w bazie danych, reprezentujący zdarzenie, będzie zawierał podstawowe parametry zdarzenia, tj:
		\begin{itemize}
			\item czas w którym nastąpiło wykrycie zdarzenia
			\item typ ataku
			\item adres źródłowy ataku
			\item adres docelowy ataku
			\item port źródłowy
			\item port docelowy
			\item użyty protokół
			\item dokładny wpis wygenerowany przez iptables
		\end{itemize}

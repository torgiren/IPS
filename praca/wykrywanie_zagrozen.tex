\chapter{Wykrywanie zagrożeń}
	\label{sec:wykrywanie}
	\section{SSH BruteForce}
		\subsection{Opis}
			Atak polega na ciągłej próbie połączenia się z~atakowanym komputerem za pomocą protokołu SSH, używając za każdym innego hasła z~listy haseł. Jeżeli hasło użytkownika, znajduje się na liście haseł atakującego, istnieje duże prawdopodobieństwo, że atakującemu uda się przy którejś próbie połączyć z~atakowanym komputerem.
		\subsection{Obrona}
			Jako obronę na ten typ ataku, zastosuję ograniczenie liczby połączeń z~usługą SSH do 5 na minutę.
			Po wykryciu większej ilości połączeń, wygenerowany zostanie komunikat z~ostrzeżeniem.
		\subsection{Implementacja}
			\footnotesize
			\verbatimtabinput{./../firewall/firewall_brute.sh}
			\normalsize
	\section{SYN Flood}
		\label{sec:syn_flood}
		\subsection{Opis}
			Atak ten jest jednym z~ataków DoS (Denial of Service) i~polega wysyłaniu dużej ilości pakietów SYN do atakowanego hosta w~celu nieumożliwienia pozostałym użytkownikom sieci skorzystanie z~atakowanej usługi.
			
			Atak ten wykorzystuje specyfikę protokołu TCP i~sposobu w~jaki protokół ten nawiązuje połączenie. Aby nawiązać połączenie komputer łączący się wysyła pakiet SYN do serwera.\\
			Serwer po otrzymaniu takiego pakietu, tworzy w~tablicy połączeń wpis ze stanem ``SYN RECEIVED'' oraz odpowiada na to żądanie pakietem SYN+ACK sygnalizując swoją gotowość do nawiązania połączenia.\\
			Następnie komputer łączący się wysyła pakiet ACK. Po takim nawiązaniu połączenia, nazywanym \textit{three\dywiz way handshake}, oba hosty przeszły w~stan połączeniowy (ESTABLISHED) i~mogą zacząć wysyłać dane.

			Atak ten polega na wysyłaniu dużej ilości spreparowanych pakietów SYN, z~podmienionymi adresami źródłowymi, do atakowanego hosta.
			W~efekcie atakowany host odpowiada pakietami SYN+ACK na adres podany jako adres źródłowy. Jeżeli jako adres źródłowy został podany aktywny host w~sieci i~otrzyma on nieoczekiwaną odpowiedź SYN+ACT, odpowie na nią pakietem RST.
			Po tej odpowiedzi atakowany host usunie wpis o~połączeniu ze swojej tablicy.\\
			Jeżeli natomiast jako adres źródłowy podamy nieaktywnego hosta w~sieci, atakowany host odpowie do niego pakietem SYN+ACT i~nie dostanie odpowiedzi, gdyż host jest nieaktywny.
			Spowoduje to, że atakowany host będzie czekał ustalony jako TIMEOUT czas, aż uzna że taki host nie istnieje i~wtedy usunie wpis z~tablicy połączeń.
			Przez ten czas, wpis jest obecny w~tablicy połączeń. Jeżeli wysłana zostanie duża ilość pakietów SYN ze zmienionymi adresami źródłowymi,
			połączenia w~stanie SYN RECEIVED wypełnią całą tablicę i~nie będą przyjmowane kolejne połączenia.\\
			W~takim przypadku, próby połączenia się z~atakowanym hostem przez zwykłych użytkowników zakończą się niepowodzeniem, gdyż host nie będzie przyjmował nowych połączeń z~powodu przepełnienia tablicy połączeń.
			Jednocześnie, tablica ta nie będzie wolna, gdyż wysyłana w~sposób ciągły fala pakietów SYN przez atakującego, wypełnia wolne miejsca po połączeniach, które zostały już usunięte z~powodu przekroczenia TIMEOUT.
			
			Nowsze wersje jądra Linux\dywiz a~posiada wbudowaną obronę przez SYN Floodem.
			Aby zasymulować starszą wersję systemu, wyłączymy mechanizm syn\_cookies:
			\footnotesize
			\begin{verbatim}
			echo "0" > /pros/sys/net/ipv4/tcp_syncoockies
			\end{verbatim}
			\normalsize
			
		\subsection{Obrona}
			Aby zabezpieczyć się przed tego typu atakami, należy limitować ilość przychodzących pakietów SYN od jednego odbiorcy.\\
			Jako domyślne wartości przyjąłem 1 połączenie przychodzące na sekundę, aktywowane po 5 pakietach.
		\subsection{Implementacja}
			\label{sec:syn_flood_impl}
			\footnotesize
			\verbatimtabinput{./../firewall/firewall_syn_flood.sh}
			\normalsize
	\section{ICMP Flood}
		\subsection{Opis}
			ICMP Flood to najczęstszy z~ataków mających na celu całkowite odcięcie dostępu do atakowanego hosta.
			Polega on na wysyłaniu bardzo dużej ilości danych. Ilość tych danych musi być większa niż przepustowość łącza atakowanego hosta.
			Następuje wtedy nasycenie pasma, i~żądania od zwykłych klientów nie są dostarczane do hosta.
			Następuje odmowa dostępu.
			
			Ataki tego tupu noszą nazwę DoS (Denial of Service) gdyż powodują odmowę dostępu do usługi.
			Jednakże, aby wykonać taki atak, atakujący musi dysponować łączem o~większej przepustowości niż atakowany host.\\
			Istnieje również odmiana ataków DoS poprzez flooda nie wymagająca większego łącza. Są to ataki DDoS (Distributed Denial of Service).
			Działają one w~myśl tej samej idei wysycania łącza atakowanego hosta, jednak uzyskiwane jest to przy użyciu wielu komputerów atakujących.
			W~takim przypadku suma przepustowości wszystkich atakujących komputerów musi być większa niż pasmo atakowanego hosta.\\
			Do takich ataków najczęściej wykorzystywane są tzw. \textit{botnet\dywiz y}, czyli sieci komputerów zainfekowanych wirusami,
			które przez większą część swojego życia na zainfekowanym komputerze nie wykazują żadnej aktywności.
			W~momencie kiedy właściciel takiego botneta chce go wykorzystać, rozsyłana jest informacja do wszystkich zainfekowanych komputerów o~celu ataku i~przeprowadzany jest atak.
		\subsection{Obrona}
			Obrona przed atakami wykorzystującymi wysycanie łącza jest niemożliwa przy użyciu netfilter.\\
			Wynika to z~faktu, że gdy host jest atakowany dużą ilością pakietów, które wysycają łącze, netfilter może odrzucać pakiety agresora dopiero gdy docierają one do firewalla,
			czyli gdy już wysycą łącze. Nie ma możliwości przy użyciu narzędzia netfilter na zapobieganiu dostarczania pakietów do naszego komputera.
	\section{ICMP Timestamp Request}
		\subsection{Opis}
			Żądanie ICMP Timestamp Request o~numerze 13, jest badaniem podania czasu serwera.
			Nie jest ono samo w~sobie atakiem, ale poznanie dokładnego czasu atakowanego hosta, może ułatwić złamanie algorytmów bazujących na generatorach liczb pseudolosowych opartych o~aktualny czas.
			
			Do serwera wysyłane jest zapytanie o~numerze typu 13, a~odpowiedź jest odsyłana w~ICMP Timestamp Reply o~numerze 14. Wiele nowoczesnych systemów w~domyślnej konfiguracji blokuje pakiety Timestamp Request.
		\subsection{Obrona}
			W~przypadku kiedy nie używamy synchronizacji czasu z~serwerem za pomocą ICMP, możemy blokować zapytania tego typu.
		\subsection{Implementacja}
			\footnotesize
			\verbatimtabinput{./../firewall/firewall_timestamp_request.sh}
			\normalsize
	\section{Skanowanie portów pakietami SYN}
		\label{sec:syn_scan}
		\subsection{Opis}
			Skanowanie portów pozwala na zbadanie atakowanego hosta pod kątem udostępnianych przez niego usług. Znając działające usługi na hoście, można dobierać odpowiednie metody.		
			
			Wysyłając pakiet SYN na skanowany port, host może zareagować na 4 sposoby:
			\begin{description}
			\item[odpowiedź SYN+ACK]\hfill \\
			oznacza ona, że port jest otwarty i~nasłuchuje na nim jakaś usługa.
			Odpowiedź SYN+ACK jest drugim pakietem wymienianym w~\textit{3-way handshake}, co pokazuje nam, że została rozpoczęta procedura nawiązywania połączenia.
			\item[odpowiedź RST+ACK]\hfill \\
			oznacza ona, że port jest zamknięty.
			Jeżeli host otrzymuje żądanie SYN na port na którym nie jest spodziewane nawiązywanie połączenia, odpowiada pakietem z~ustawioną flagą reset, która informuje o~zerwaniu połączenia (w~tym przypadku o~zerwaniu próby połączenia z~portem)
			\item[odpowiedź ICMP error message]\hfill \\
			oznacza ona, że port jest filtrowany na firewallu przez reguły REJECT, które generują odpowiedź do klienta o~niemożności połączenia się z~portem.
			Taka odpowiedź daje nam informacje, że na skanowanym hoście jest aktywny firewall, natomiast nie daje nam wiedzy czy na danym porcie działa jakaś usługa - połączenia mogą być filtrowane ze względu na IP źródłowe.
			\item[brak odpowiedzi]\hfill \\
			może być oznaką zarówno błędów sieci i~zgubienia pakietów (w~przypadku ograniczenia liczby retransmisji), bądź obecności firewalla i~filtrowania pakietów metodą DROP.
			Pakiety takie zostają upuszczone i~nie zostaje wysyła odpowiedź do hosta skanującego.
			Scenariusz błędów sieci jest zwykle mniej prawdopodobny, dlatego brak odpowiedzi najczęściej świadczy o~obecności firewalla na skanowanym hoście.
			\end{description}
			
			Skanowanie portów metodą pakietów SYN jest bardzo podobne do metody ataku SYN Flood.
			Istnieją jednak pewne aspekty odróżniające te dwa działania, a~mianowicie:
			\begin{itemize}
			\item ilość wysyłanych pakietów - w~SYN Flood wysyłamy bardzo dużo pakietów, aby zapełnić tablicę połączeń, w~skanowaniu wystarczy wysłać po jednym pakiecie na każdy badany port
			\item adres źródłowy pakietów - w~SYN Flood podawaliśmy nieistniejący adres źródłowy, aby połączenie trwało w~oczekiwaniu na odpowiedź,
			natomiast w~skanowaniu portów, podajemy swój adres jako adres źródłowy, abyśmy mogli odebrać i~zinterpretować odpowiedź serwera.
			\end{itemize}
		\subsection{Obrona}
			Metoda obrony przed skanowaniem portów metodą SYN jest taka sama jak w~przypadku SYN Flood, gdyż tak samo otrzymujemy dużą ilość pakietów SYN.
		\subsection{Implementacja}
			\label{sec:syn_scan_impl}
			Patrz: \ref{sec:syn_flood_impl}: \nameref{sec:wykrywanie}/\nameref{sec:syn_flood}/\nameref{sec:syn_flood_impl} na stronie \pageref{sec:syn_flood_impl}.
	\section{Skanowanie portów funkcjami systemowymi}
		\subsection{Opis}
			Skanowanie portów funkcjami systemowymi, wykorzystuje oferowaną przez system operacyjny obsługę połączeń sieciowych.
			Nie dają one możliwości preparowania pakietów, dlatego w~przypadku natrafienia na otwarty port przeprowadzana jest kompletna procedura \textit{3-way\dywiz handshake} jak również nie ma możliwości zmiany adresów źródłowych ani innych wartości w~ramkach pakietu.
			Jednak, jest to jedyna metoda która może być wykonana na komputerze bez dostępu do konta administratora.
		\subsection{Obrona}
			Funkcje systemowe aby nawiązać połączenie wysyłają pakiety SYN, dlatego obrona przed tego typu skanowaniem jest taka sama jak przy wysyłaniu spreparowanych pakietów SYN.
		\subsection{Implementacja}
			Patrz: \ref{sec:syn_scan_impl}: \nameref{sec:wykrywanie}/\nameref{sec:syn_scan}/\nameref{sec:syn_scan_impl} na stronie \pageref{sec:syn_scan_impl}.
	\section{Skanowanie portów pakietami ACK}
		\subsection{Opis}
			Skanowanie portów pakietami ACK wykorzystuje specyfikację protokołu TCP, który na nieoczekiwany pakiet z~flagą ACK odpowiada pakietem RST.

			Wysyłając spreparowany pakiet TCP z~ustawioną flagą ACK, sprawiamy że system operacyjny na atakowanym hoście, nie jest w~stanie go dopasować do żadnego istniejącego połączenia i~odsyła pakiet z~ustawioną flagą RST do atakującego hosta informując w~ten sposób o~nieprawidłowościach.\\
			Po otrzymaniu odpowiedzi RST mamy informację, że atakowany nie filtruje na firewallu danego portu.

			Metoda ta nie daje informacji czy na danym porcie jest uruchomiona jakaś usługa, a~jedynie czy dany port jest filtrowany na firewallu.
			W~przypadku filtrowania portu, nie dostaniemy żadnej odpowiedzi, gdyż pakiet zostanie upuszczone.
			W~przypadku braku firewalla, wysłana zostanie odpowiedź RST.
		\subsection{Obrona}
			Aby obronić się przed takim atakiem, powinniśmy blokować wszystkie pakiety mające ustawioną flagą ACK oraz będące interpretowane jako nowe połączenia zamiast ustanowione.
		\subsection{Implementacja}
			\footnotesize
			\verbatimtabinput{./../firewall/firewall_ack_scan.sh}
			\normalsize
	\section{Spoofing z~sieci wewnętrznej}
		\subsection{Opis}
			Ataki wykorzystujące spoofing polegają na podszywaniu się pod innego użytkownika sieci.
			Mają one na celu zafałszowanie informacji o~atakującym.\\
			Wykonując atak SYN FLOOD (patrz: \ref{sec:syn_flood}), używaliśmy spoofingu w~celu opóźnienia usuwania wpisów z~tablicy.
			Jednak spoofingu można użyć np: do wykonania podejrzanego skanowania portów jako inny użytkownik sieci.
			Sprawi to, że administrator sieci, bazując na zapisach adresu IP z~którego zostało wykonane skanowanie portów bądź inna próba ataku, skupi się na jego analizie, co może przysporzyć nieprzyjemności podszywanemu użytkownikowi sieci.\\
			Innym celem takiego ataku, może być wywołaniu kilku oczywistych ataków z~komputerów ofiar, tak, aby skupić uwagę administratora na tamtych wydarzeniach i~w~tym czasie wykonać cichy atak na serwer.\\
			Wyróżniamy dwa główne typu spoofingu:
			\begin{description}
				\item[IP Spoofing]\hfill \\
					Polega on na preparowaniu pakietów podmieniając adres źródłowy IP.
					Efektem takiego ataku jest interpretowanie takich pakietów przez serwer jako pakietów wysyłany przez komputer ofiary.
					Minusem tej metody jest to, że odpowiedzi generowane przez serwer odsyłane są do komputera ofiary, ponieważ jej  komputer odpowiada na zapytania ARP od serwera.\\
					Ominięciem tej niedogodności jest ustawienie ręczne adresu IP identycznego z~adresem ofiary.
					Wtedy mamy możliwość nawiązywania pełnych połączeń z~serwerem jako komputer ofiary.
				\item[ARP Spoofing]\hfill \\
					Atak ten polega na podmianie wpisów w~pamięci podręcznej protokołu ARP.
					
					Protokół ARP(ang. Address Resolution Protocol) jest protokołem pozwalającym na dopasowanie adresów MAC do adresów IP.
					Komunikacja wykorzystująca adresy IP wykonywana jest w~3 warstwie modelu OSI - warstwie sieciowej.
					Jednakże, przełączniki działają w~warstwie 2 - warstwie łącza danych, dlatego wysyłając pakiet IP, należy go opakować w~ramkę łącza danych zawierającą adres MAC odbiorcy i~nadawcy.
					Aby użytkownik nie musiał podawać tej wartości, opracowany został protokół ARP.\\
					Zasada działania protokołu ARP:\\
					Jeżeli system operacyjny chce wysłać pakiet do odbiorcy o~podanym adresie IP, sprawdza pamięć podręczną w~poszukiwaniu adresu MAC odbiory.
					W~sytuacji w~której adres ten zostanie tam znaleziony, zostaje on wykorzystany.
					W~przeciwnym przypadku system wysyła do wszystkich komputerów w~sieci zapytanie o~adres MAC komputera o~poszukiwanym IP.
					Na tą prośbę odpowiada poszukiwany komputer, odpowiadając pakietem podpisanym swoim adresem MAC.
					Po otrzymaniu takiego pakietu, adres MAC zostaje zapisany w~pamięci podręcznej i~wykorzystany do wysłania pierwotnej wiadomości.
					
					Podatnością tego protokołu jest interpretacja odpowiedzi ARP bez wcześniejszego zgłaszania zapotrzebowania na adres.
					Atak ARP Spoofing polega na wysłaniu do atakowanego hosta informacji ARP Reply zawierającej adres źródłowy podszywanego komputera oraz MAC adres atakującego.
					Atakowany host po otrzymaniu takiej informacji zapisze ją w~pamięci podręcznej, bądź, jeśli już istniał wpis dla takiego adres IP - nadpisze starą informację nowa.
					Spowoduje to, że wysyłane pakiety z~serwera do podszywanego komputera będą podpisywane adresem MAC atakującego, gdyż jego MAC znajduje się w~pamięci podręcznej atakowanego hosta.
					Atakujący host powinien mieć ustawioną opcję ip\_forward aby pakiet docelowo wysłany do podszywanego komputera, docierający do atakującego, mógł zostać wysłany dalej aby mógł osiągnąć swój cel.
					Jednak przechodząc przez komputer atakującego, daje on możliwość przechwycenia znajdujących się w~nim informacji a~nawet ingerencję w~jego zawartość.
			\end{description}
		\subsection{Obrona}
			Najlepszą metodą obrony przed spoofingiem jest zbudowanie bazy znanych hostów sieci wewnętrznej i~umieszczenie jej w~kontenerze IPSet.
			Najlepiej nadającym do się do tego tupu zadań kontenerem jest kontener typu \textit{MACIPMAP}.
			Przechowuje on pary adres MAC\dywiz adres IP.
			Następnie dla każdego pakietu przychodzącego do serwera należy sprawdzić czy jego para IP\dywiz MAC znajduje się w~bazie.
			Jeżeli nie znajduje się, może to oznaczać jedną z~dwóch możliwości:
			\begin{itemize}
				\item jest to nowy host w~sieci
				\item nastąpiła próba spoofingu
			\end{itemize}
			Aby wykluczyć pierwszą możliwość, administrator może sprawdzić czy MAC adres pakietu znajduje się w~bazie.
			Jeżeli znajduje się, to oznacza to, że pakiet został spreparowany i~prawdopodobnie wysłany w~celu przeprowadzenia ataku.
		\subsection{Implementacja}
			\footnotesize
			\verbatimtabinput{./../firewall/firewall_spoofing_wew.sh}
			Zakładamy że interface eth0 jest naszym interfejsem do sieci wewnętrznej a~eth1 do sieci zewnętrznej
			\normalsize
	\section{Spoofing z zewnątrz}
		\subsection{Opis}
			Ataki tego typu polegają na podszywaniu się pod hosty z sieci wewnętrznej przy połączeniach z sieci zewnętrznej.
			Może to prowadzić do prowadzić do wykonania ataku DDoS przy użyciu adresu broadcast. 
			Wysyłając pakiety z ustawionym adresem źródłowym na IP atakowanego hosta, IP docelowym na broadcast oraz docelowym adresem MAC na adres serwera, zostaną one przekazane przez serwer na adres broadcast, po czym każdy host w sieci odpowie na taki pakiet do hosta wskazanego przez adres źródłowy.
			Efektem tego może być kilkudziesięciokrotne zwiększenie liczby otrzymywanych przez atakowanego hosta pakietów do liczby wysyłanych pakietów przez atakującego.
		\subsection{Obrona}
			Aby obronić się przed tego typu atakami, należy filtrować wszystkie pakiety przychodzące na interfejs zewnętrzny posiadające adresy źródłowe należące do klas prywatnych.
		\subsection{Implementacja}
			\footnotesize
			\verbatimtabinput{./../firewall/firewall_spoofing_zew.sh}
			Zakładamy że interface eth0 jest naszym interfejsem do sieci wewnętrznej a~eth1 do sieci zewnętrznej
			\normalsize
	\section{Wirusy}
		\subsection{Opis}
			Obecność wirusów w sieci może prowadzić do infekowania kolejnych komputerów, a w przypadku utworzenia tzw. \textit{botnet}-u, przeprowadzenie ataku.
			Aktywność większości wirusów charakteryzuje się wysyłaniem dużej ilości poczty elektronicznej.
		\subsection{Obrona}
			Należy wychwytywać i blokować zbyt duży ruch poczty elektronicznej.
			Jednakże, należy tutaj uważać, na tzw. \textit{false-positive} dopasowania, czyli uznanie normalnego ruchu pocztowego jako ruchu spamowego.
			Sytuacja taka może nastąpić podczas wysyłania aplikacji do pracy, gdzie następuje wysyłanie dużej ilości wiadomości do pracodawców.
		\subsection{Implementacja}
			\footnotesize
			\verbatimtabinput{./../firewall/firewall_wirusy.sh}
			\normalsize


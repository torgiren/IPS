\documentclass[a4paper]{article}
\usepackage{polski}
\usepackage[utf8]{inputenc}
\author{Marcin TORGiren Fabrykowski}
\title{Wstępna specyfikacja systemu IPS}
\begin{document}
\maketitle
\newpage
\tableofcontents
\newpage
\section{Opis problemu}
\subsection{Co to jest IPS}
System IPS (Intrusion Prevention System) jest to system wykrywania i blokowanie ataków sieciowych.
Jego zadaniem jest analiza ruchu sieciowego wchodzącego do serwera oraz przechodzącego przez niego, oraz odpowiednie reagowanie w przypadku wykrycia **podejrzanych** zachowań
\subsection{Schemat działania}
\begin{enumerate}
\item Analiza ruchu sieciowego
\item Wykrycie zachowań pasujących do reguł bezpieczeństwa
\item Reakcja na niebezpieczne zachowanie
\item Poinformowanie administratora o próbie ataku
\item Zapisanie podjętych działań w bazie danych
\end{enumerate}
\section{Analiza ruchu sieciowego}
Jako systemu analizującego ruch sieciowy wykorzystam pakiet Netfilter konfigurowany za pomocą iptables.
\section{Wykrywanie zagrożeń}
\subsection{Ataki typu SSH BruteForce}
\subsubsection{Opis}
Atak ten polega na ciągłej próbie połączenia się z usługą SSH z różnymi hasłami.
\subsubsection{Obrona}
Zastosuję ograniczenie liczby połączeń z usługą SSH w ciągu minuty do 4.\\
Dodatkowo przy liczbie połączeń przekraczającej 10 na godzinę zostanie wygenerowane ostrzeżenie.
\subsubsection{Implementacja}
\end{document}

